%%% Layout
\documentclass[11pt]{article}
\usepackage{fancyhdr}
\pagestyle{headings}
\usepackage{fullpage}


%%% Math
\usepackage{amsmath}
\usepackage{amssymb}
\usepackage{esint}
\usepackage{indentfirst}
\usepackage{gensymb}
\usepackage{mathtools}
\usepackage{listings}

%%% Graphics
\usepackage{graphicx}
\usepackage[subrefformat=parens,labelformat=parens]{subfig}

%%% Bibliography
\usepackage[verbose]{cite}
\bibliographystyle{unsrt}
\usepackage{url}

%%% Mark-up
\usepackage{color} % text color
%\usepackage{soul} % mark up effects
%\usepackage{todonotes}

%%% Title
\lhead{AA 543 Winter 2017 HW\#5}
\chead{}
\rhead{Crews \& Thomas, HW\#5}
\cfoot{\thepage}
\title{AA 543 Winter 2017 HW\#5}
\author{D. W. Crews and W.R. Thomas}
\date{\today}

\begin{document}
\maketitle

\section{Definition of problem}
Solve numerically the 2D Euler equations for the transonic flow over a NACA 0012 airfoil using the Jameson scheme (Jameson et al., AIAA 1981).  

\noindent The following conditions are given:
\begin{itemize}
	\item Angle of attack $\alpha = 0 \deg$
	\item Free-stream Mach number $M_\infty = 0.85$
\end{itemize}

\noindent These further free-stream conditions are defined given an altitude of 10 km as per the Standard Atmosphere Table 1976.
\begin{itemize}
	\item $\gamma \approx 1.4$
	\item Density $\rho_\infty = 0.414 \; \text{kg/m}^3$ 
	\item Temperature $T_\infty = 223.3 $ K
	\item Pressure $p_\infty = 26.5 $ kPa
	\item Sound speed $c_\infty = \sqrt{\gamma \frac{p_\infty}{\rho_\infty}} = 299 \approx 300 $ m/s
	\item Velocity $|\vec{u}_\infty| = M_\infty c_\infty = 255$ m/s. 
	
	Therefore $u_\infty = |\vec{u}_\infty| \cos\alpha = 255$ m/s and  $v_\infty = |\vec{u}_\infty| \sin\alpha = 0$ m/s.
	\item Energy $E_\infty = e + \frac{|\vec{u}_\infty|^2}{2}$ where $e = \frac{p_\infty}{\rho_\infty (\gamma -1)} $. The free-steam energy is then $E_\infty = 193 $ J
\end{itemize}

\section{Development of the numerical scheme}
	\subsection{Grid}
	Grid created in Homework \#3.
	
	(Copy calcs for cell center point, area and cell wall normals from notes)
	
	\subsection{Normalization (??)}
	(Normalize to free-stream values?)
	
	\subsection{Initial conditions}
	Initial conditions will be the same as the free-stream values, for all points in the domain.
		\begin{align}
		U_{i,j}^0 = \begin{bmatrix}
		\rho_\infty \\ \rho_\infty u_\infty \\ \rho_\infty v_\infty \\ \rho_\infty E_\infty
		\end{bmatrix}
		\end{align}
	
	\subsection{Boundary conditions}
		\subsubsection{Outer boundary}
		The outer boundary conditions will be a subsonic ($M_\infty=0.85$) laminar flow in the $\hat{x}$ direction.  Because the boundary is circular, and some regions will be inflow, and some outflow, the boundary condition applied will depend on the orientation of the boundary cell normals to the fluid velocity $\vec{u} \cdot \hat{n}$.
		\subsubsection*{Subsonic inflow, $\vec{u} \cdot \hat{n} > 0$}
		We have the following requirements
			\begin{align}
			\vec{u}_{nB} = \vec{u}_{i,M+1} \cdot \hat{n}_{i,M} = \frac{R_{i,M+1}^+ \cdot \hat{n}_{i,M} + R_{i,M}^- \cdot \hat{n}_{i,M}}{2} \\
			\vec{u}_{i,M+1} \cdot \hat{l}_{i,M} = \vec{u}_\infty \cdot \hat{\ell}_{i,M} \\
			s_{i,M+1} = s_\infty 
			\end{align}
		\subsubsection*{Subsonic outflow, $\vec{u} \cdot \hat{n} < 0$}
		
		\subsubsection{Air foil boundary}
		Along the air foil, there will be a no slip boundary, and normal velocity $u_n = 0$. Density, pressure, energy and transverse velocity will be extrapolated linearly from the interior.
			\begin{align}
			U_{i,0-1} = \begin{bmatrix}
			\tilde{\rho} \\ \tilde{\rho} \tilde{\tilde{u}} \\ \tilde{\rho} \tilde{\tilde{v}} \\ \tilde{\rho} \tilde{E}
			\end{bmatrix}_{i,0-1}
			\end{align}
		Where the $\tilde{\tilde{u}}$ and $\tilde{\tilde{v}}$ contain the extrapolated transverse velocity and an equal but opposite normal velocity to the cell inside the boundary
			\begin{align}
			\tilde{\tilde{u}}_{i,-1} = (-[u\hat{x} + v\hat{y}]_{i,0} \hat{n}_{i,-1} + [\tilde{u}\hat{x} + \tilde{v}\hat{y}]_{i,-1} \hat{\ell}_{i,-1}) \cdot \hat{x} \\
			\tilde{\tilde{v}} _{i,-1} = (-[u\hat{x} + v\hat{y}]_{i,0} \hat{n}_{i,-1} + [\tilde{u}\hat{x} + \tilde{v}\hat{y}]_{i,-1} \hat{\ell}_{i,-1}) \cdot \hat{y}
			\end{align}
		The normal unit vectors $\hat{n}_{i,j}$ points in the same direction as the boundary wall normal vector 
			\begin{align}
			\hat{n}_{i,-1} = \vec{s}_{i,j}^J/\Omega_{i,j}
			\end{align}
		and the transverse unit vector $\hat{\ell}_{i,j}$ is the normal vector rotated $90 \degree$ in the clockwise direction. 
			\begin{align}
			\hat{\ell}_{i,-1} =  R(90 \degree) \hat{n}_{i,-1} = \begin{bmatrix} 0 & 1 \\ -1 & 0 \end{bmatrix} \hat{n}_{i,-1}
			\end{align}
	
		For extrapolated values in ghost cells for density, transverse velocity, and energy, the following equation is used (here $q$ is an arbitrary scalar variable)
			\begin{align}
			\tilde{q}_{i,-1} = ax_{i,-1} + by_{i,-1} + c
			\end{align}
		Where the coefficients $a, b$ and $c$ are found using a linear least squares fit on $q_{i,0}, q_{i,1}$ and $q_{i,2}$.
		
		\subsubsection{Boundaries along O-mesh slice}
		These boundaries will be periodic. There is no need for ghost cells as
			\begin{align}
			U_{-1,j} = U_{N-1,j}; \quad U_{N+1,j} = U_{0,j}
			\end{align}
	\subsection{Spacial discretization of 2D Euler equations with Jameson scheme}
	(copy from notes)
	
	\subsection{Temporal discretization with Runge-Kutta 4 step time integration scheme}
	(copy from notes)
	
	\subsection{Determining convergence to steady state condition}

\section{Results \& Analysis}	
		
	
\subsection{Bonus}

	
	

\pagebreak
\appendix
\section{Code}
	\subsection{2d\_euler\-jameson.cc}
%		\lstinputlisting[language=C++]{../code/2d_euler_jameson.cc}
		
	

\end{document}

